\documentclass[12pt]{article}
\usepackage[left=3cm, right=3cm, bottom=2.5cm, top=2.5cm]{geometry}
\usepackage{setspace}
\usepackage[T1]{fontenc}
\usepackage{times}
\usepackage{booktabs}
\usepackage{rotating}
\usepackage{graphicx}
\usepackage[section]{placeins} %Placeins.sty keeps floats `in their place', preventing them from floating past a "\FloatBarrier" command into another section.  To use it, declare "\usepackage{placeins}" and insert "\FloatBarrier" at places that floats should not move past, perhaps at every "\section".  
\usepackage[large, bf]{caption}
\usepackage[FIGTOPCAP]{subfigure}
\usepackage{pdfpages}
\usepackage{palatino}
\usepackage{pdflscape}
\usepackage{textcomp}
\usepackage{longtable}
\usepackage{nicefrac}
\usepackage{adjustbox}	% to adjust the size of objects to fit into a page
\usepackage[hyphens]{url}


% Citing 
%\usepackage{natbib}
%\bibpunct{(}{)}{;}{a}{,}{,}

%\def\citeapos#1{\citeauthor{#1}'s (\citeyear{#1})}


%% Citing with footnotes
% replace \cite with \autocite (and vice versa if we want to go back to apalike eferences
%\usepackage[style=verbose,backend=bibtex]{biblatex}
\usepackage[notes, backend=biber]{biblatex-chicago}
\bibliography{references}


%zero spacing between references
%\usepackage{bibspacing}
%\setlength{\bibspacing}{\baselineskip}

%%-----------------------------------------------------------------
%%Header
%\usepackage{fancyhdr}
%\fancyhf{}
%\fancyhead[C]{\textit{Preliminary and Incomplete}}
%\fancyfoot[C]{\thepage}
%\renewcommand\headrulewidth{0pt}
%\pagestyle{fancy}
%%-----------------------------------------------------------------

%\onehalfspacing
%\doublespacing

\usepackage{amsmath, amsfonts, amssymb, amsthm}

\usepackage{mathpazo} %Use Palotino fonts
\parskip 0ex  %Vertical distance between paragraphs, in "ex"s
\parindent 20pt

%\usepackage{harvard}
%\bibliographystyle{apsr}
%\bibliographystyle{dcu}

\usepackage[pdftex]{hyperref}
\hypersetup{colorlinks, citecolor=black, filecolor=blue, linkcolor=blue, urlcolor=blue}

%\makeatletter
%\renewcommand{\subsubsection}{\@startsection
%{subsubsection} %the name
%{4} %the level
%{0pt} %the indent 
%{1ex} %the before skip
%{1ex} %the after skip
%{\itshape}}
%\makeatother

\newtheorem{theorem}{Theorem}
\newtheorem{lemma}{Lemma}
\newtheorem{proposition}{Proposition}
\newtheorem{corollary}{Corollary}
\newtheorem{prediction}{Prediction}
\newtheorem{case}{Special Case}

\newenvironment{proofAlt}[1][Proof]{\begin{trivlist}
\item[\hskip \labelsep {\bfseries #1}]}{\end{trivlist}}
\newenvironment{definition}[1][Definition]{\begin{trivlist}
\item[\hskip \labelsep {\bfseries #1}]}{\end{trivlist}}
\newenvironment{example}[1][Example]{\begin{trivlist}
\item[\hskip \labelsep {\bfseries #1}]}{\end{trivlist}}

\newenvironment{remark}[1][Remark]{\begin{trivlist}
\item[\hskip \labelsep {\bfseries #1}]}{\end{trivlist}}
\def\urltilda{\kern -.15em\lower .7ex\hbox{\~{}}\kern .04em}

\renewcommand{\thesubfigure}{(\Alph{subfigure})}

%%%%%%%%%%%%%%%%%%%%%%%%%%%%%%%%%%
% SPACING
%%%%%%%%%%%%%%%%%%%%%%%%%%%%%%%%%%

\usepackage{titlesec}

\titlespacing*{\section}{0pt}{1.5ex plus 1ex minus .2ex}{0.8ex plus .2ex}
\titlespacing*{\subsection}{0pt}{1.2ex plus 1ex minus .2ex}{0.8ex plus .2ex}


%%%%%%%%%%%%%%%%%%% 
% LANDSCAPE
%%%%%%%%%%%%%%%%%%%%%%%%%%%
\usepackage{lscape}

%%% TABLES

\usepackage{subfig}