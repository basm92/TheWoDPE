    Historically, democratic nation states in Western Europe emerged in the long nineteenth century out of a desire among elites to curb the power of the monarch, and subject their decisions to control. Such systems tended to evolve into systems where executives were responsible for much of the law-making, and the power of the King or monarch was further diminished (and in some countries, monarchies were nullified) up to the point of formality. Decision-making was, however, to be approved by a parliament, consisting of elected delegates by the part of the population deemed eligible to vote. 

    After 1848, following the revolutions in France and Germany, such a change also took place in the Netherlands. The King at the time, Willem II, ceded power to parliament and ordered the principal liberal politician of the time, J.R. Thorbecke, to draft a revised Constitution, thereby enshrining the regime of parliamentary democracy. At the outset, however, the system was still oligarchic: most politicians were either landed aristocrats, or theologians and professors, belonging to the country's intellectual and cultural elite. 

    Under the pressure of prominent liberals, and perhaps more importantly, the increasing presence of socialism, the Netherlands embarked on a trajectory of abolishing most electoral restrictions in place, leading to an increase in the enfranchised population in various steps, and culminating in universal suffrage, approved by parliament in 1919. These gradual increases in the enfranchised population were accompanied by various trends: first, the country saw a large increase in religious tensions and religious segregation. Second, politicians from different social backgrounds and milieus were elected in the lower house. The country thus transitioned from a society ruled by traditional elites to a society where the representative bodies were much more like the general population in terms of social background. \autocite{van1983toegang} Third, the country's politicians laid the groundwork for the welfare state by adopting laws that boosted public education, instigating redistribution and increasing government spending. 

    The first trend is \textit{pillarization}, which is specific to the Dutch setting: the increase in religious tensions and segregation in the country. The 1848 Constitutional revision brought to power a liberal government, which set about to emancipate the country's Catholic and Jewish citizens, starting symbolically with the reestablishment of the episcopal hierarchy in the Netherlands. This decision was met with considerable opposition, spearheaded by some of the most prominent Orthodox Protestant intellectuals, who responded by offering a petition with more than 50,000 signatures to the King. The King in turn refused to reject the petition, which he had been dictated by the government to do, following which the government resigned. \autocite{oud1961honderd} Liberals subsequently accused the King of alleged partiality to Protestantism, violating the separation of church and state dictated by the novel constitution. A subsequent law reaffirming the separation of church and state mitigated tensions, but for the remainder of the century, politicians from both Protestant and Catholic confessions set out to arrange as many aspects of life as possible within their pillar, leaving comparatively little space for central government intervention. \autocite{van2013eerste}

    Secondly, following increased democratization, electoral politics became competitive, forcing established political parties and newspapers to pay attention not only to incumbent candidates and politicians from traditional elites, but also to newcomers with political ambitions from backgrounds less associated with the country's elite. The country's upper house, on the contrary, remained a very exclusive institution, limiting its membership to the highest taxpayers of the country. The country's executive positions also remained exclusive to technocrats or other individuals belonging to the country's elites: although perhaps related to the expertise required, in many cases, personal links between Ministers and the King are documented. \autocite{secker1991ministers}

    Thirdly, the country saw a modest increase in public expenditures and redistribution following protests and increasing popularity of socialist politicians. Compared to other Western countries, social expenditures in the Netherlands were at a similar level, but rather low in absolute terms, and failed to reach a breakthrough until well after the Second World War. [REF] Oftentimes, proposals to give the state an increasing role in the economy were met with fierce opposition, especially from Catholic politicians, who opted for a privatized form of care organized on the basis of the country's religious 'pillars', although at various points in time, near unanimity was reached to amend particularly pressing issues, such as child labor, public housing, and education. \autocite{van2013eerste}

    When comparing the Netherlands to other Western European countries,  the pattern of suffrage extensions seems to have been very much in line with various other Western European countries. Most Western European countries seem to have enacted constitutional reforms, if not in the early 19th century, then after the 1848 riots or the Paris Commune. The majority of countries then embarked on a similar trajectory culminating in universal suffrage shortly after World War I, with only very few countries granting universal suffrage at once \autocite{caramani2017elections}. The trajectory with respect to gender discrimination was also similar. A French court decided against suffrage of women in 1885 \autocite{przeworski2009conquered}, whereas in the Netherlands, attempts by feminist activist Aletta Jacobs to secure female suffrage were sabotaged in 1887 by a constitutional amendment.\footnote{This 1887 amendment did however, imply a substantial extension of suffrage to men.} Countries that instigated universal male suffrage around the same time as the Netherlands (1917) include Luxembourg, the United Kingdom, Sweden, and many others. \autocite{caramani2017elections}

    When comparing the Netherlands to other countries in terms of public expenditures and redistribution, roughly the same view arises: the Netherlands seem to slightly lag behind some of their neighbouring countries, only catching up after World War II \autocite{lindert2004growing}. In particular, social transfers as a percentage of GDP remained low relative to other countries, stagnating at about 2.5\% of GDP before 1930, at a level comparable to France and Italy at a time. Other parts of public expenditures were similarly small, with the notable exception of expenditures on education, which the government agreed to finance in 1917. School enrollment was on a level comparable to Belgium, France and Norway. Poor relief as a government expenditure was also a very marginal part of GDP, at least up until 1880. \autocite{van2000eenheiddstaat, lindert2004growing}

    In sum, the Netherlands seem to undergo trends in parallel to various other (Western) European countries. Government expenditures steadily increased over time, but remain modest by today's standards before the outbreak of World War I. Furthermore, the country moved towards universal suffrage, which it achieved in 1919\footnote{The law instigating universal suffrage was approved in 1919, and the first parliamentary election in which every citizen aged 25 or older could vote took place in 1922.}. The aspect that stands out, however, is pillarization. With the exception of Germany, the virtually all Western European countries were far more religiously homogeneous than the Netherlands. However, German Protestants were for the larger part Lutherans, whereas Dutch Protestants were by and large exclusively Calvinists, and religious tensions in Germany were not nearly as aggravated as in the Netherlands, and the country's governance was not organized around religious affiliation. This paper attempts to investigate the role of politicians in the processes of pillarization, changing representation, and the enactment of a welfare state by investigating wealth of parliaments and politicians over time, and as a function of political affiliation.

    Many researchers allege that politicians are inclined to pursue their own interests instead of the interests of their constituents. \autocite{lizzeri2004did, duggan2017political, corvalan2020political} Politicians more similar to the general population might be more inclined to pursue policies that favor redistribution, intervention, and progressive taxation. Finding out whether, and to what extent, politicians are similar to the population provides us with evidence as to whether this mechanism is at play or not: the personal wealth of politicians can influence the trajectories of public spending, and democratization. Furthermore, another important motive for finding out how wealthy politicians are is provided by the literature on political selection. Many empirical and theoretical studies have argued that following a relaxation in eligibility and suffrage restrictions historically employed by many countries, the political arena should  become more diverse in terms of social origin, gender, and many other aspects.\autocite{besley1997economic, besley2005political, bernini2018race} In addition, several theoretical models, as well as empirical studies, e.g.  show that government policy is responsive to changes in the characteristics of politicians. \autocite{meltzer1981rational, besley2011educated, chattopadhyay2004women, hayo2014political} Hence, the continuity, or change, in the composition of politicians can influence political decision-making. 

    Whereas the present-day literature has investigated various factors, as of yet, the role of personal wealth of politicians has never been elucidated. For starters, it is unknown whether and to what extent the wealth of the political elite resembles the general population. Secondly, it is unknown what the consequences are of a wealth gap between politicians and their constituents. In this paper, we first show the evolution of wealth in parliament over time. We use carefully compiled lists of all members per parliament, including politicians who took the place of politicians who resigned, died, or left their position because of other reasons. We identify the wealthiest and poorest members of parliament. Furthermore, we contrast the lower house to the upper house. It is well known that the upper house remained much more exclusive than the lower house, the entry of which was (in theory) open to any male candidate. \autocite{van1983toegang} We also distinguish between lower house members, provincial executives and ministers. In other words, this paper makes a step in the direction of finding out what the role of politicians is by asking how large the gap is between politicians and the electorate in terms of wealth and several other characteristics, and how this various over time when various suffrage extensions are enacted (an aspect which the Netherlands shares with many other countries) and with respect to political affiliation (an aspect in which the Netherlands is rather unique). 